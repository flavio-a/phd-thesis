% !TEX root = ../phd-thesis.tex

\chapter{A Comparison of Program Logics}\label{ch:sil}
In this chapter, we consider three known triple-based program logics, namely Hoare Logic (HL), Incorrectness Logic (IL) and Necessary Conditions (NC). We characterize their validity conditions in term of over or under-approximation of forward and backward semantics. First, this allows us to identify the absence of one combination, and thus to define a new program logic, called Sufficient Incorrectness Logic (SIL), for it. Second, this guides a thorough comparison of the four validity conditions, highlighting analogies and differences between them.

The content of this chapter is based on~\cite{ABGL24}.

\section{Taxonomy}
\begin{figure}[t]
	\centering
	\begin{tabular}{c|ccc}
		                   & Forward                   &                                        & Backward                                            \\[3pt]
		\hline             &                           &                                        &                                                     \\[-5pt]
		\quad Over\quad\   & \quad Hoare Logic         & $\xleftrightarrow{\qquad\simeq\qquad}$ & Necessary Conditions                                \\ [5pt]
		\quad Under\quad\  & \quad Incorrectness Logic &                                        & \textcolor{ACMBlue}{Sufficient Incorrectness Logic}
	\end{tabular}
	\caption{The taxonomy of triple-based logics. They are sorted depending on whether they use forward or backward semantics and over or under-approximation. HL and NC are equivalent ($\simeq$). SIL is the new logic introduced here.}
	\label{fig:sil:taxonomy}
\end{figure}

As discussed in Sections~\ref{sec:bg:hl} and \ref{sec:bg:il}, the validity conditions of HL and IL are defined as over and under-approximation of the forward semantics $\fwsem{\cdot}$. However, other program logics cannot be naturally described in terms of $\fwsem{\cdot}$. To this end, we consider a backward semantics $\bwsem{\cdot}$ defined as the opposite relation of the forward semantics, that is

\begin{equation}
	\bwsem{\regr} \sigma' \eqdef \{ \sigma \svert \sigma' \in \fwsem{\regr} \sigma \} \label{eq:sil:bwsem-definition}
\end{equation}

\noindent or, equivalently,

\begin{equation}
	\sigma \in \bwsem{\regr} \sigma' \iff \sigma' \in \fwsem{\regr} \sigma  \label{eq:sil:bwsem-sigma-sigma'}
\end{equation}

\noindent and we lift this definition of backward semantics to set of states by union.
Intuitively, the forward semantics $\fwsem{\regr} P$ denotes the set of all possible output states of $\regr$ when execution starts from a state in $P$ (and $\regr$ terminates).
Instead, the backward semantics $\bwsem{\regr} Q$ denotes the set of all input states that can lead to a state in $Q$.

The backward semantics can also be characterized compositionally, similarly to the forward one:

\begin{lemma}\label{lmm:sil:bwsem-calculus}
	For any regular commands $\regr,\regr_1,\regr_2\in \Reg$, the following equalities hold:
	\[
	\bwsem{\regr_1; \regr_2} = \bwsem{\regr_1} \circ \bwsem{\regr_2} \qquad\qquad
	\bwsem{\regr_1 \regplus \regr_2} = \bwsem{\regr_1} \cup \bwsem{\regr_2} \qquad\qquad
	\bwsem{{\regr^\kstar}} = \bigcup\limits_{n \ge 0} \bwsem{\regr}^n
	\]
\end{lemma}

Using $\bwsem{\cdot}$, we characterize NC with the validity condition $\bwsem{\regr} Q \subseteq P$:

\begin{prop}[NC as backward over-approximation]\label{prop:sil:nc}
	Given a postcondition $Q$ for the program $\regr$, any possible necessary precondition $\underline{P}$ for $Q$ satisfies
	\[
	\bwsem{\regr} Q \subseteq \underline{P}
	\]
\end{prop}

Note that this is not an entirely new understanding of NC (it was already hinted in the work that introduced them, and an analogous characterization appeared in~\cite[§6.3]{ZK22} in terms of weakest precondition), but the explicit use of the backward semantics in our contest enables a more streamlined comparison with other logics in Section~\ref{sec:sil:comparison}.

We organize the validity conditions of HL, IL and NC in the taxonomy in Figure~\ref{fig:sil:taxonomy}. We classify logics depending on (1) whether the condition is expressed in terms of forward or backward semantics and (2) whether it is an over or an under-approximation.
This naturally sparks the question on what does the backward under-approximation condition mean and whether a logic for it has been developed.

\paragraph{Backward under-approximation and Lisbon Triples}
At POPL'19 in Lisbon, D. Dreyer and R. Jung suggested that P. O'Hearn should look at bug-finding in terms of a logic for proving the presence of faults (as reported in~\cite{OHearn20,ZDS23}).
However, the proposed model of triples did not fit well with a key feature of Pulse, a bug-catching tool developed mainly by J. Villard at Meta, namely its ability to drop the analysis of some program paths, for which IL provides a sound logical foundation instead.
The idea of such ``Lisbon'' triples is that \emph{for any initial state satisfying the pre, there exists some execution trace leading to a final state satisfying the post} and it can be dated back to Hoare's calculus of possible correctness~\cite{Hoare78}, even if no form of approximation was considered there.
Lisbon triples were then briefly discussed in~\cite[\S 5]{MOH21} and \cite[\S 3.2]{LRVBDO22} under the name \emph{backwards under-approximate triples}. They were also one of the motivations for OL~\cite{ZDS23} and are identified as certain OL triple.

Ideally, given an incorrectness specification, the goal of backward under-approximation would be to report to programmers all dangerous input states that lead to bugs.
However, all the above proposals are designed according to the forward semantics of programs and thus are best suited to infer postconditions starting from some given precondition: in general, their backward analysis may require the instantiation of their consequence rules with some ingenious guess.
To tackle this issue, we introduce Sufficient Incorrectness Logic (SIL) as a proof system for Lisbon triples with backward\hyp{}oriented rules.

\begin{figure}
	\centering
	\begin{tabular}{c|c@{\qquad}c@{\qquad}c@{\qquad}c@{\qquad}c}
		Logic                       &
		HL \cite{Hoare69}           &
		IL \cite{OHearn20}          &
		NC \cite{CCL11}             &
		OL \cite{ZDS23}             &
		SIL
		\\[2pt] \hline &&&&& \\[-10pt]
		Triples                     &
		$\overtriple{P}{\regr}{Q}$  &
		$\undertriple{P}{\regr}{Q}$ &
		$\necctriple{P}{\regr}{Q}$  &
		$\oltriple{P}{\regr}{Q}$    &
		$\angletriple{P}{\regr}{Q}$
	\end{tabular}
	\caption{Summary of the notation for different program logics}
	\label{fig:sil:notation-summary}
\end{figure}

Since in this chapter we deal with different kinds of program logics, a summary of the notation for the various triples is reported in Figure~\ref{fig:sil:notation-summary}.

\section{Sufficient Incorrectness Logic}
Interestingly, IL and SIL share the possibility to drop disjuncts, which was in fact one of the leading motivations for the introduction of IL to increase scalability and make the methods effective: the difference is that IL drops disjuncts in the postconditions, while SIL in the preconditions, a feature that is illustrated in Section~\ref{sec:separation-sil-derivation}.

\section{Relations Among Logics}\label{sec:sil:comparison}
We summarize the validity of different kinds of triples in the

\begin{definition}[Triples validity]\label{def:sil:validity}
	For any program $\regr$ and sets of states $P,Q$ we let the following:

	\smallskip

	\begin{tabular}{llll}
		\textbf{HL triples:}  & $\overtriple{P}{\regr}{Q}$  & is valid if & $\fwsem{\regr} P \subseteq Q$; \\
		\textbf{IL triples:}  & $\undertriple{P}{\regr}{Q}$ & is valid if & $\fwsem{\regr} P \supseteq Q$; \\
		\textbf{NC triples:}  & $\necctriple{P}{\regr}{Q}$  & is valid if & $\bwsem{\regr} Q \subseteq P$; \\
		\textbf{SIL triples:} & $\angletriple{P}{\regr}{Q}$ & is valid if & $\bwsem{\regr} Q \supseteq P$.
	\end{tabular}
\end{definition}

By using the validity conditions in Definition~\ref{def:sil:validity}, we can characterize HL, IL, NC, and SIL according to (i) whether the condition is expressed in terms of forward or backward semantics, and (ii) whether it is an over- or an under-approximation (Figure~\ref{fig:sil:taxonomy}).
Indeed, in HL and NC it is safe to enlarge the respective target set (i.e., $Q$ and $P$) so we do categorize them as over-approximations.
Similarly, in IL and SIL it is safe to shrink their target sets, so we do classify them as under-approximations.
Furthermore, from the validity conditions, we can immediately derive the consequence rules for each logic: for HL and SIL we are allowed to weaken $Q$ and strengthen $P$, while for IL and NC we can do the opposite, i.e. the direction of the consequence rules are determined by the diagonals in Figure~\ref{fig:sil:taxonomy}.
Notwithstanding the fact that NC and IL share the same consequence rule, we will show that NC and IL are not related, whereas NC is tightly connected to HL.
We obtain that an NC triple $\necctriple{P}{\regr}{Q}$ is valid \emph{if and only if} the HL triple $\overtriple{\lnot P}{\regr}{\lnot Q}$ is valid.
%
By duality, one might expect a similar connection to hold between IL and SIL, but we show this is not the case.

\section{Separation Sufficient Incorrectness Logic}

\section{Conclusions}
