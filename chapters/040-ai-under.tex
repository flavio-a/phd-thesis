% !TEX root = ../phd-thesis.tex

\chapter{Under-Approximation Abstract Domains}\label{ch:limits-underapprox-domains}
In this chapter, we try to use Abstract Interpretation as a basis for under-approximation analysis. In principle, the over-approximation theory could be dualized in an order-theoretic sense to obtain results for under-approximation. However, in this chapter we show that it's not so simple. Particularly, \emph{basic transfer functions} (the semantics of basic constructs of the language) are not dualized: therefore, the dual of an abstract domain that "behaves well" with respect to basic transfer functions may not enjoy the same property.

We first point out some intuitive reasons that break the symmetry between over and under-approximation. Then, building on these observations, we formally derive some negative results showing that it is not possible to define Galois connection-based under-approximation abstract domains in a large class of instances. More in details, we assume that (i)~abstract analyses should return non-trivial results for large classes of programs and (ii)~to justify the convenience of the abstract analysis, the abstract domain should be significantly “smaller” than the concrete powerset. Under these assumptions, we prove that there is no under-approximation abstract domain able to analyse programs encoding certain classes of basic transfer functions.

The content of this chapter is based on~\cite{ABG22,ABG24}.

\fromhere
\section{Overview}
A shortened version of the introduction, with related work, the general idea and the examples.

\section{Non-emptying functions}

\section{Integer domains}

\section{General infinite concrete domains}

\section{General finite concrete domains}

\section{Conclusions}
