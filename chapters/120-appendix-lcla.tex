% !TEX root = ../phd-thesis.tex

\chapter{\texorpdfstring{$\LCLA$}{LCL} supplementary materials}\label{ch:app:lcla}
This appendix contains technical details of proofs and examples for Chapter~\ref{ch:lcla}.

\begin{proof}[Proof of Theorem~\ref{th:lcla:soundness-ext}, extensional soundness of $\LCLA$]
	First we remark that points (1) and (3) implies point (2):
	\begin{align*}
		\alpha(Q) & \le \alpha(\denot{\regr} P)     & [\text{(1) and monotonocity of }\alpha] \\
		          & \le \denot{\regr}^{A} \alpha(P) & [\text{soundness of }\denot{\regr}^{A}] \\
		          & = \alpha(Q)                     & [\text{(3)}]
	\end{align*}
	So all the lines are equal, in particular $\alpha(Q) = \alpha(\denot{\regr} P)$.
	The proof is then by induction on the derivation tree of $\lcl{A}{P}{\regr}{Q}$, but we only have to prove (1) and (3) because of the observation above.

	\proofcase{\lclrule{transfer}}
	\noindent (1) It follows since $\denot{\expe} P \le \denot{\expe} P$.

	\noindent (3) It follows from the local completeness hypothesis $\complete{A}{P}{\denot{\expe}}$: $\denot{\expe}^{A} \alpha(P) = \alpha(\denot{\expe} \gamma \alpha(P)) = \alpha(\denot{\expe} P)$.

	\proofcase{\lclrule{relax}}
	\noindent (1) By inductive hypothesis we know $Q' \le \denot{\regr} P'$, and together with the other two hypotheses of the rule we have $Q \le Q' \le \denot{\regr} P' \le \denot{\regr} P$.

	\noindent (3) We remark that $\alpha(P) = \alpha(P')$ if and only if $A(P) = A(P')$ by injectivity of $\gamma$ in a GI, and that $P' \le P \le A(P')$ implies $A(P) = A(P')$. Thus the hypotheses of the rule gives $\alpha(P) = \alpha(P')$ and $\alpha(Q) = \alpha(Q')$. Point (3) then follow by the inductive hypothesis $\denot{\regr}^{A} \alpha(P') = \alpha(Q')$:
	\[
	\denot{\regr}^{A} \alpha(P) = \denot{\regr}^{A} \alpha(P') = \alpha(Q') = \alpha(Q)
	\]

	\proofcase{\lclrule{seq}} (1) $Q \le \denot{\regr_2} R \le \denot{\regr_2} (\denot{\regr_1} P) = \denot{\regr1; \regr_2} P$, where the inequalities follow from inductive hypotheses and monotonicity of $\denot{\regr_2}$.

	\noindent (3) We recall that $\denot{\regr_1; \regr_2}^{A} \le \denot{\regr_2}^{A} \denot{\regr_1}^{A}$.
	\begin{align*}
		\alpha(Q) & \le \alpha(\denot{\regr_1; \regr_2} P)                & [\text{(1) and monotonicity of }\alpha] \\
		          & \le \denot{\regr_1; \regr_2}^{A} \alpha(P)            & [\text{soundness of }\denot{\regr}^{A}] \\
		          & \le \denot{\regr_2}^{A} \denot{\regr_1}^{A} \alpha(P) & [\text{recalled above}]                 \\
		          & = \denot{\regr_2}^{A} \alpha(R)                       & [\text{inductive hp}]                   \\
		          & = \alpha(Q)                                           & [\text{inductive hp}]
	\end{align*}
	So all the lines are equal, in particular $\denot{\regr_1; \regr_2}^{A} \alpha(P) = \alpha(Q)$.

	\proofcase{\lclrule{join}}
	\noindent (1) By inductive hypotheses, $Q_1 \le \denot{\regr_1} P$ and $Q_2 \le \denot{\regr_2} P$. Hence $Q_1 \lor Q_2 \le \denot{\regr_1} P \lor \denot{\regr_2} P = \denot{\regr_1 \oplus \regr_2} P$.

	\noindent (3) We observe that
	\begin{align*}
		\denot{\regr_1 \oplus \regr_2}^{A} \alpha(P) & = \alpha \denot{\regr_1 \oplus \regr_2} \gamma \alpha(P)                                   \\
		                                             & = \alpha (\denot{\regr_1} \gamma \alpha (P) \sqcup \denot{\regr_2} \gamma \alpha (P))      \\
		                                             & = \alpha \denot{\regr_1} \gamma \alpha (P) \sqcup \alpha \denot{\regr_2} \gamma \alpha (P) \\
		                                             & = \denot{\regr_1}^{A}\alpha (P) \sqcup \denot{\regr_2}^{A} \alpha (P)
	\end{align*}
	where we used additivity of $\alpha$.
	Recalling that by inductive hypotheses, $\alpha(Q_1) = \denot{\regr_1}^{A} \alpha(P)$ and $\alpha(Q_2) = \denot{\regr_2}^{A} \alpha(P)$, we get
	\begin{align*}
		\alpha(Q_1 \lor Q_2) & = \alpha(Q_1) \sqcup \alpha(Q_2)                                     & [\alpha\text{ is additive}]   \\
		                     & = \denot{\regr_1}^{A} \alpha(P) \sqcup \denot{\regr_2}^{A} \alpha(P) & [\text{inductive hypotheses}] \\
		                     & = (\denot{\regr_1 \oplus \regr_2}^{A}) \alpha(P)                     & [\text{observation above}]
	\end{align*}

	\proofcase{\lclrule{rec}}
	\noindent (1) First, we show that $\denot{\regr^{\kstar}} R \le \denot{\regr^{\kstar}} P$ using the inductive hypothesis $R \le \denot{\regr} P$:
	\begin{align*}
		\denot{\regr^{\kstar}} R & = \bigsqcup_{n \ge 0} \denot{\regr}^n R                 & [\text{definition}]                    \\
		                         & \le \bigsqcup_{n \ge 0} \denot{\regr}^n \denot{\regr} P & [\text{inductive hp on all operands}]  \\
		                         & = \bigsqcup_{n \ge 1} \denot{\regr}^n P                 & [\text{renaming the index variable }n] \\
		                         & \le \bigsqcup_{n \ge 0} \denot{\regr}^n P               & [\text{adding an element to the lub}]  \\
		                         & = \denot{\regr^{\kstar}} P                              & [\text{definition}]
	\end{align*}
	Now we show (1):
	\begin{align*}
		Q & \le \denot{\regr^{\kstar}} (P \lor R)                      & [\text{inductive hp}]                         \\
		  & = \denot{\regr^{\kstar}} P \lor \denot{\regr^{\kstar}} R   & [\text{additivity of }\denot{\regr^{\kstar}}] \\
		  & \le \denot{\regr^{\kstar}} P \lor \denot{\regr^{\kstar}} P & [\text{observation above}]                    \\
		  & = \denot{\regr^{\kstar}} P
	\end{align*}

	\noindent (3)
	\begin{align*}
		\denot{\regr^{\kstar}}^{A} \alpha(P) & \le \denot{\regr^{\kstar}}^{A} \alpha(P \lor R) & [\text{monotonicity of }\denot{\regr^{\kstar}}^{A} \alpha] \\
		                                     & = \alpha(Q)                                     & [\text{inductive hp}]                                      \\
		                                     & \le \alpha(\denot{\regr^{\kstar}} P)            & [\text{(1) and monotonicity of }\alpha]                    \\
		                                     & \le \denot{\regr^{\kstar}}^{A} \alpha(P)        & [\text{soundness of }\denot{\regr^{\kstar}}^{A}]
	\end{align*}
	Hence all the lines are equal, and in particular $\denot{\regr^{\kstar}}^{A} \alpha(P) = \alpha(Q)$.

	\proofcase{\lclrule{iterate}}
	\noindent (1) We observe that $P = \denot{\regr}^0 P \le \denot{\regr^{\kstar}} P$. Moreover, by inductive hypothesis $Q \le \denot{\regr} P$ so also $Q \le \denot{\regr^{\kstar}} P$. So by definition of lub $P \lor Q \le \denot{\regr^{\kstar}} P$.

	\noindent (3) First we show by induction on $n \ge 1$ that $\alpha \denot{\regr}^n \gamma \alpha(P) \le \denot{\regr}^{A} \alpha(P)$.
	The base case $n = 1$ is trivial recalling that $\denot{\regr}^{A} = \alpha \denot{\regr} \gamma$. So suppose it is true for $n$ and let us show it for $n + 1$:
	\begin{align*}
		\alpha \denot{\regr}^{n+1} \gamma \alpha(P) & \le \alpha \denot{\regr} \gamma \alpha \denot{\regr}^n \gamma \alpha(P) & [\gamma \alpha \le \id]                                                                        \\
		                                            & \le \denot{\regr}^{A} \denot{\regr}^{A} \alpha(P)                       & [\text{inductive hp } \alpha \denot{\regr}^n \gamma \alpha(P) \le \denot{\regr}^{A} \alpha(P)] \\
		                                            & = \denot{\regr}^{A} \alpha(Q)                                           & [\text{inductive hp }\lcl{A}{P}{\regr}{Q}]                                                     \\
		                                            & \le \denot{\regr}^{A} \alpha(P)                                         & [\text{hypothesis of the rule }Q \le A(P)]
	\end{align*}
	For the last inequality, we used that in any GC $Q \le A(P) = \gamma \alpha(P)$ if and only if $\alpha(Q) \le \alpha(P)$.
	Now we can prove the following chain of inequalities:
	\begin{align*}
		\denot{\regr^{\kstar}}^{A} \alpha(P) & = \alpha \bigsqcup_{n \ge 0} \denot{\regr}^n \gamma \alpha(P)                                                               & [\text{definition}]                                               \\
		                                     & = \bigsqcup_{n \ge 0} \alpha \denot{\regr}^n \gamma \alpha(P)                                                               & [\alpha \text{ is additive}]                                      \\
		                                     & = \alpha \denot{\regr}^0 \gamma \alpha(P) \sqcup \left( \bigsqcup_{n \ge 1} \alpha \denot{\regr}^n \gamma \alpha(P) \right) & [\text{splitting the lub}]                                        \\
		                                     & \le \alpha \denot{\regr}^0 \gamma \alpha(P) \sqcup \left( \bigsqcup_{n \ge 1} \denot{\regr}^{A} \alpha(P) \right)           & [\text{shown above by induction}]                                 \\
		                                     & = \alpha \denot{\regr}^0 \gamma \alpha(P) \sqcup \denot{\regr}^{A} \alpha(P)                                                & [\text{lub of constant terms}]                                    \\
		                                     & = \alpha(P) \sqcup \denot{\regr}^{A} \alpha(P)                                                                              & [\denot{\regr}^0 = \id \text{ and }\alpha \gamma \alpha = \alpha] \\
		                                     & = \alpha(P) \sqcup \alpha(Q)                                                                                                & [\text{inductive hp}]                                             \\
		                                     & = \alpha(P \lor Q)                                                                                                          & [\alpha \text{ is additive}]                                      \\
		                                     & \le \alpha (\denot{\regr^{\kstar}} P)                                                                                       & [\text{(1) and monotonicity of }\alpha]                           \\
		                                     & \le \denot{\regr^{\kstar}}^{A} \alpha(P)                                                                                    & [\text{soundness of }\denot{\regr^{\kstar}}^{A}]
	\end{align*}
	Thus all the lines above are equal, and in particular $\denot{\regr^{\kstar}}^{A} \alpha(P) = \alpha(P \lor Q)$.
\end{proof}

This technical lemma is used in the following proofs.
\begin{lemma}\label{th:lemma-AA'}
	If $A' \preceq A$ then $A = A A' = A' A$
\end{lemma}
\begin{proof}
	Fix a concrete element $c \in C$. Since $A' \preceq A$ we have $c \le A'(c) \le A(c)$. Applying $A$, by monotonicity we get $A(c) \le AA'(c) \le AA(c) = A(c)$, where the last equality is idempotency of $A$. This means $A = AA'$.
	Now consider $A' A(c)$. Since $A$ is a closure operator $A' A(c) \le A (A' A(c))$. But we just showed $A A' (A(c)) = A (A(c)) = A(c)$. Lastly, since $A'$ is a closure operator too, $A(c) \le A' A(c)$. Hence $A(c) \le A' A(c) \le A(c)$, so $A(c) = A' A(c)$.
\end{proof}
We point out that, by injectivity of $\gamma$, this also means $\alpha \gamma' \alpha' = \alpha$.

\begin{proof}[Proof of Theorem~\ref{th:lcla:soundness-rule-refine}, extensional soundness of rule \lclrule{refine\mbox{-}ext}]
	We recall that the intuitive premise $A \denot{\regr}^{A'} A(P) = A(Q)$ of the rule formally is $\alpha \gamma' \denot{\regr}^{A'} \alpha' A(P) = \alpha(Q)$.
	Since the proof of Theorem~\ref{th:lcla:soundness-ext} is by induction, we extend it by just proving the new inductive case. We also remark that we only need to prove points~(1) and~(3), since they imply point~(2).

	\proofcase{\lclrule{refine\mbox{-}ext}}
	\noindent (1) It's the same as point (1) of extensional soundness applied to $\lcl{A'}{P}{\regr}{Q}$, since this conclusion does not depend on the abstract domain.

	\noindent (3)
	\begin{align*}
		\alpha(Q) & \le \alpha(\denot{\regr} P)                                             & [\text{(1) and monotonicity of }\alpha]  \\
		          & \le \denot{\regr}^{A} \alpha(P)                                         & [\text{soundness of } \denot{\regr}^{A}] \\
		          & = \alpha \denot{\regr} \gamma \alpha(P)                                 & [\text{definition}]                      \\
		          & = \alpha \gamma' \alpha' \denot{\regr} \gamma' \alpha' \gamma \alpha(P) & [\text{Lemma~\ref{th:lemma-AA'}}]        \\
		          & = \alpha \gamma' \denot{\regr}^{A'} \alpha' A(P)                        & [\text{definition}]                      \\
		          & = \alpha(Q)                                                             & [\text{hypothesis of the rule}]
	\end{align*}
	Hence all the lines are equal; in particular $\denot{r}^{A} \alpha(P) = \alpha(Q)$.
\end{proof}

\begin{proof}[Proof of Proposition~\ref{th:lcla:refine-int-sound}, extensional soundness of rule \lclrule{refine\mbox{-}int}]
	We show that the hypotheses of \lclrule{refine\mbox{-}int} implies those of \lclrule{refine\mbox{-}ext}. This means than whenever we can apply the former we could also apply the latter, so that Theorem~\ref{th:lcla:soundness-rule-refine} ensures extensional soundness.

	The first two hypotheses $\lcl{A'}{P}{\regr}{Q}$ and $A' \preceq A$ are shared among the two rules, so we only have to show $\alpha \gamma' \denot{\regr}^{A'} \alpha' A(P) = \alpha(Q)$. We recall that $\lcl{A'}{P}{\regr}{Q}$ implies $Q \le \denot{\regr} P$ by extensional soundness.
	\begin{align*}
		\alpha(Q) & \le \alpha(\denot{\regr} P )                                & [Q \le \denot{\regr} P \text{ and monotonicity of }\alpha] \\
		          & \le \denot{\regr}^{A} \alpha(P)                             & [\text{soundness of }\denot{\regr}^{A}]                    \\
		          & = \alpha \denot{\regr} A(P)                                 & [\text{definition}]                                        \\
		          & = \alpha \gamma' \alpha' \denot{\regr} A' A(P)              & [\text{Lemma~\ref{th:lemma-AA'}}]                          \\
		          & = \alpha \gamma' \denot{\regr}^{A'} \alpha' A(P)            & [\text{definition}]                                        \\
		          & \le \alpha \gamma' \denot{\regr}^{\sharp}_{A'} \alpha' A(P) & [\denot{\regr}^{A'} \le \denot{\regr}^{\sharp}_{A'}]       \\
		          & = \alpha(Q)                                                 & [\text{Last hypothesis of the rule}]
	\end{align*}
	Hence all the lines are equal, and in particular $\alpha \gamma' \denot{\regr}^{A'} \alpha' A(P) = \alpha(Q)$.
\end{proof}

\begin{proof}[Proof of Proposition~\ref{th:lcla:refine-pre-sound}, extensional soundness of rule \lclrule{refine\mbox{-}pre}]
	As in the above proof or Proposition~\ref{th:lcla:refine-int-sound}, we show that the hypotheses of \lclrule{refine\mbox{-}pre} implies those of \lclrule{refine\mbox{-}ext}.

	The first two hypotheses $\lcl{A'}{P}{\regr}{Q}$ and $A' \preceq A$ are shared among the two rules, so we only have to show $\alpha \gamma' \denot{\regr}^{A'} \alpha' A(P) = \alpha(Q)$. We recall that $\lcl{A'}{P}{\regr}{Q}$ implies by extensional soundness (1) $Q \le \denot{\regr} P$ and (3) $\denot{\regr}^{A'} \alpha'(P) = \alpha'(Q)$.
	\begin{align*}
		\alpha(Q) & \le \alpha(\denot{\regr} P )                   & [Q \le \denot{\regr} P \text{ and monotonicity of }\alpha] \\
		          & \le \denot{\regr}^{A} \alpha(P)                & [\text{soundness of }\denot{\regr}^{A}]                    \\
		          & = \alpha \denot{\regr} A(P)                    & [\text{definition}]                                        \\
		          & = \alpha \denot{\regr} A'(P)                   & [\text{hp of the rule}]                                    \\
		          & = \alpha \gamma' \alpha' \denot{\regr} A'(P)   & [\text{Lemma~\ref{th:lemma-AA'}}]                          \\
		          & = \alpha \gamma' \denot{\regr}^{A'} \alpha'(P) & [\text{definition}]                                        \\
		          & = \alpha \gamma' \alpha'(Q)                    & [\text{extensional soundness (3)}]                         \\
		          & = \alpha(Q)                                    & [\text{Lemma~\ref{th:lemma-AA'}}]
	\end{align*}
	Hence all the lines are equal, and in particular $\alpha \gamma' \denot{\regr}^{A'} \alpha' A(P) = \alpha(Q)$.
\end{proof}

\begin{sidewaysfigure}
	\[
	\scriptsize
	\infer[(\mathsf{seq})]
	{ ($**$) }
	{
		\infer[(\mathsf{transfer})]
		{ \lcl{\Oct}{\code{y} \in \{2; 100\} \land \code{x} = \code{y}}{\code{y := y - 1}}{ \code{y} \in \{1; 99\} \land \code{x} - 1 = \code{y} } }
		{ \complete{\Oct}{\code{y} \in \{2; 100\} \land \code{x} = \code{y}}{\denot{\code{y := y - 1}}} }
		&
		\infer[(\mathsf{transfer})]
		{ \lcl{\Oct}{ \code{y} \in \{1; 99\} \land \code{x} - 1 = \code{y} }{\code{x := x - 1}}{\code{y} \in \{1; 99\} \land \code{x} = \code{y}} }
		{ \complete{\Oct}{\code{y} \in \{1; 99\} \land \code{x} - 1 = \code{y}}{\denot{\code{x := x - 1}}} }
	}
	\]
	\[
	\small
	\infer[(\mathsf{iterate})]
	{ ($*$) }
	{
		\infer[(\mathsf{seq})]
		{ \lcl{\Oct}{R_2}{\regr_i}{\code{y} \in \{1; 99\} \land \code{x} = \code{y}}
			\qquad (\code{y} \in \{1; 99\} \land \code{x} = \code{y}) \le A(R_2) }
		{
			\infer[(\mathsf{transfer})]
			{ \lcl{\Oct}{R_2}{\code{(x > 1)?}}{\code{y} \in \{2; 100\} \land \code{x} = \code{y}} }
			{ \complete{\Oct}{R_2}{\denot{\code{(x > 1)?}}}}
			&
			\infer[(\mathsf{seq})]
			{ \lcl{\Oct}{\code{y} \in \{2; 100\} \land \code{x} = \code{y}}{\code{y := y - 1; x := x - 1}}{\code{y} \in \{1; 99\} \land \code{x} = \code{y}} }
			{ ($**$) }
		}
	}
	\]
	\[
	\small
	\infer[(\mathsf{seq})]
	{\lcl{\Oct}{R}{\regr_2}{Q}
		%					\qquad \Int_{\neq 0} \preceq \Int \qquad \Int(P) = \Int_{\neq 0}(P)
	}
	{
		\infer[(\mathsf{transfer})]
		{\lcl{\Oct}{R}{\code{x := y}}{R_2}}{ \complete{\Oct}{R}{\denot{\code{x := y}}} }
		&
		\infer[(\mathsf{seq})]
		{\lcl{\Oct}{R_2}{\regr_i^{\kstar}\code{; (x <= 1)?}}{Q}}
		{
			\infer[(\mathsf{iterate})]
			{\lcl{\Oct}{R_2}{\regr_i^{\kstar}}{R_2 \lor (\code{y} \in \{1; 99\} \land \code{x} = \code{y})}}
			{ ($*$) }
			&
			\infer[(\mathsf{transfer})]
			{\lcl{\Oct}{R_2}{\code{(x <= 1)?}}{Q}}{ \complete{\Oct}{R_2}{\denot{\code{(x <= 1)?}}} }
		}
	}
	\]
	\caption{Derivation of $\lcl{\Oct}{R}{\regr_2}{Q}$ for Example~\ref{ex:lcla:refine-pre-usefulness}.}
	\label{fig:app:ex-refine-pre-derivation-2}
\end{sidewaysfigure}

\begin{example}[Details of Example~\ref{ex:lcla:refine-pre-usefulness}.]
	The full derivation of the triple $\lcl{\Oct}{R}{\regr_2}{Q}$ is shown in Figure~\ref{fig:app:ex-refine-pre-derivation-2}, rotated and split to fit the page. We call the command iterated with the Kleene star $\regr_i \eqdef \code{(x > 1)?; y := y - 1; x := x - 1}$, and we let $R_2 \eqdef (\code{y} \in \{1; 2; 100\} \land \code{x} = \code{y})$. We also used the logical implication $R_2 \implies (\code{y} \in \{1; 99\} \land \code{x} = \code{y})$, both explicitly and implicitly in the equivalence $R_2 \lor (\code{y} \in \{1; 99\} \land \code{x} = \code{y}) \iff R_2$.
\end{example}

\begin{example}\label{ex:app:refine-pre-incomplete-2-appendix}
	Similarly to Example~\ref{ex:lcla:refine-pre-incomplete}, we present another triple which is sound but cannot be proved in $\LCLA$ extended with \lclrule{refine\mbox{-}pre}. The key difference is that this triple does not rely on divergence, but only on actual (im)precisions in the abstract domain.

	Consider the concrete domain $C = \pow(\setZ)$ of integers, the abstract domain $\Int$ of intervals, the concrete initial states $P = \{ -1, 1 \}$ and commands
	\begin{align*}
		\regr_1 & \eqdef \code{x != 0?} \\
		\regr_2 & \eqdef \code{x >= 0?}
	\end{align*}
	Then, the triple $\lcl{\Int}{P}{\regr_1; \regr_2}{\{ 1 \}}$ is sound but cannot be proved in $\LCLA$ extended with \lclrule{refine\mbox{-}pre}.

	Following the same line of reasoning in Example~\ref{ex:lcla:refine-pre-incomplete}, we observe that all strict subset $P' \subset P$ are such that $\Int(P') \subset P$, and the same property holds for all refinements $A' \preceq \Int$. Again, this means that we cannot apply \lclrule{relax} to change $P$: to do it, we would need a $P' \subset P$ such that $P \subseteq A'(P')$.

	Let $f_1 = \denot{\regr_1}$ and $f_2 = \denot{\regr_2}$. Observe that in the concrete semantics $f_1(P) = P$ and $f_2(P) = \{ 1 \}$.
	Inspecting the logic, to prove the triple $\lcl{\Int}{P}{\regr_1; \regr_2}{\{ 1 \}}$ we can only apply three rules: \lclrule{relax}, \lclrule{refine\mbox{-}pre} or \lclrule{seq}.
	To apply rule \lclrule{relax} we would need either an under\hyp{}approximation $P'$ of $P$ with the same abstraction, that does not exist by the observation above, or an over\hyp{}approximation of $Q$, that would be unsound since $Q = f(P)$. Hence we cannot apply \lclrule{relax}. Suppose to apply \lclrule{refine\mbox{-}pre}: any $A'$ used in the rule should satisfy $A' \preceq \Int$ and $A'(P) = \Int(P)$. As we pointed out above, we cannot apply \lclrule{relax} even after the domain refinement. Therefore, the only rule that can be applied is \lclrule{seq}. To do that, we must prove two triples $\lcl{A'}{P}{\regr_1}{R}$ and $\lcl{A'}{R}{\regr_2}{Q}$ for some $R$.

	Irrespective of how we prove the first triple, by soundness (Theorem~\ref{th:lcla:soundness-ext}) we have $R \subseteq f_1(P) = P$ and $A'(R) = A'(f_1(P)) = A'(P)$. If $R \subset P$ then $A'(R) \subset P \subseteq A'(P)$, which is a contradiction. So $R = P$.
	Now we should prove a triple $\lcl{A'}{P}{\regr_2}{Q}$, but this is impossible since, by soundness, this would imply local completeness of $\denot{\regr_2} = f_2$ on $P$ in $A'$, that does not hold:
	\begin{align*}
		A' f_2(P) & = A'(\{ 1 \})                                 \\
		          & \subseteq \Int(\{ 1 \}) = \{ 1 \}             \\
		          & \subset \{ 0, 1 \} = f_2(\Int(P)) = f_2 A'(P) \\
		          & \subseteq A' f_2 A'(P)
	\end{align*}

	\begin{figure*}[t]
		\[
		\text{\small
		\infer[(\mathsf{refine\mbox{-}int})]
		{\lcl{\Int}{P}{\regr_1; \regr_2}{Q}}
		{\infer[(\mathsf{seq})]
		{\lcl{\Int_P}{P}{\regr_1; \regr_2}{Q} \quad \Int_P \preceq \Int \quad \Int(\denot{\regr}^{\sharp}_{\Int_P}(\Int(P))) = \Int(Q)}
		{
		\infer[(\mathsf{transfer})]
		{\lcl{\Int_P}{P}{\code{x != 0?}}{P}}{ \complete{\Int_P}{P}{\denot{\code{x != 0?}}} }
		& &
		\infer[(\mathsf{transfer})]
		{\lcl{\Int_P}{P}{\code{x >= 0?}}{Q}}{ \complete{\Int_P}{P}{\denot{\code{x >= 0?}}} }
		}
		}
		}
		\]
		\caption{Derivation of $\lcl{\Int}{P}{\regr}{Q}$ for Example~\ref{ex:app:refine-pre-incomplete-2-appendix}.}\label{fig:app:ex-refine-int-incomplete-2}
	\end{figure*}

	Observe that, if we add \lclrule{refine\mbox{-}int} to the proof system, we can use it to change the domain to one where we can express $P$ (for instance, the concrete domain $\pow(\setZ)$ or the refinement $\Int_P \eqdef \Int \cup \{ P \}$) to prove the triple by applying \lclrule{seq} and then \lclrule{transfer} on both subtrees, as shown in Figure~\ref{fig:app:ex-refine-int-incomplete-2}.
	The two local completeness proof obligations are trivially satisfied because $P \in \Int_P$, and the hypothesis of \lclrule{refine\mbox{-}int} is verified:
	\begin{align*}
		\Int( \denot{\regr}^{\sharp}_{\Int_P} \Int(P)) & = \Int( \denot{\regr_2}^{\Int_P} \denot{\regr_1}^{\Int_P} \Int(P))  \\
		                                               & = \Int( \denot{\regr_2}^{\Int_P} \denot{\code{x != 0?}^{\Int_P}} P) \\
		                                               & = \Int( \denot{\code{x >= 0?}} P)                                   \\
		                                               & = \Int( [1; 1] ) = \Int(Q)
	\end{align*}
\end{example}

The following lemma is used in the subsequent proof.
\begin{lemma}\label{lmm:app:comp-abs-monotone-domain}
	Let $\regr \in \Reg$ be a regular command. If $A \preceq A'$ then $\gamma \denot{\regr}^{\sharp}_{A} \alpha \le \gamma' \denot{\regr}^{\sharp}_{A'} \alpha'$.
\end{lemma}
\begin{proof}
	The proof is by structural induction on $\regr$.

	\proofcase{($\regr = \expe$)}
	$\gamma \denot{\regr}^{\sharp}_A \alpha = \gamma \alpha \edenot{\expe} \gamma \alpha = A \edenot{\expe} A \le A' \edenot{\expe} A' = \gamma' \denot{\regr}^{\sharp}_{A'} \alpha'$.

	\proofcase{($\regr_1; \regr_2$)}
	\begin{align*}
		\gamma \denot{\regr_1; \regr_2}^{\sharp}_A \alpha & = \gamma \denot{\regr_2}^{\sharp}_A \denot{\regr_1}^{\sharp}_A \alpha                           & [\text{definition}]          \\
		                                                  & = \gamma \denot{\regr_2}^{\sharp}_A \alpha \gamma \denot{\regr_1}^{\sharp}_A \alpha             & [\alpha \gamma = \id_A]      \\
		                                                  & \le \gamma' \denot{\regr_2}^{\sharp}_{A'} \alpha' \gamma' \denot{\regr_1}^{\sharp}_{A'} \alpha' & [\text{inductive hp}]        \\
		                                                  & = \gamma' \denot{\regr_2}^{\sharp}_{A'} \denot{\regr_1}^{\sharp}_{A'} \alpha'                   & [\alpha' \gamma' = \id_{A'}] \\
		                                                  & = \gamma' \denot{\regr_1; \regr_2}^{\sharp}_{A'} \alpha'                                        & [\text{definition}]
	\end{align*}

	\proofcase{($\regr_1 \oplus \regr_2$)}
	\begin{align*}
		\gamma \denot{\regr_1 \oplus \regr_2}^{\sharp}_A \alpha & = \gamma (\denot{\regr_2}^{\sharp}_A \alpha \sqcup \denot{\regr_1}^{\sharp}_A \alpha)                                          & [\text{definition}]           \\
		                                                        & = \gamma (\alpha \gamma \denot{\regr_2}^{\sharp}_A \alpha \sqcup \alpha \gamma \denot{\regr_1}^{\sharp}_A \alpha)              & [\alpha \gamma = \id_A]       \\
		                                                        & = \gamma \alpha (\gamma \denot{\regr_2}^{\sharp}_A \alpha \sqcup \gamma \denot{\regr_1}^{\sharp}_A \alpha)                     & [\alpha \text{ is additive}]  \\
		                                                        & \le A (\gamma' \denot{\regr_2}^{\sharp}_{A'} \alpha' \sqcup \gamma' \denot{\regr_1}^{\sharp}_{A'} \alpha')                     & [\text{inductive hp}]         \\
		                                                        & \le A' (\gamma' \denot{\regr_2}^{\sharp}_{A'} \alpha' \sqcup \gamma' \denot{\regr_1}^{\sharp}_{A'} \alpha')                    & [A \le A']                    \\
		                                                        & = \gamma' (\alpha' \gamma' \denot{\regr_2}^{\sharp}_{A'} \alpha' \sqcup \alpha' \gamma' \denot{\regr_1}^{\sharp}_{A'} \alpha') & [\alpha' \text{ is additive}] \\
		                                                        & = \gamma' (\denot{\regr_2}^{\sharp}_{A'} \alpha' \sqcup \denot{\regr_1}^{\sharp}_{A'} \alpha')                                 & [\alpha' \gamma' = \id_{A'}]  \\
		                                                        & = \gamma' \denot{\regr_1 \oplus \regr_2}^{\sharp}_{A'} \alpha'                                                                 & [\text{definition}]
	\end{align*}

	\proofcase{($\regr^{\kstar}$)}
	First we prove by induction on $n \ge 0$ that $\gamma (\denot{\regr}^{\sharp}_{A})^n \alpha \le \gamma' (\denot{\regr}^{\sharp}_{A'})^n \alpha'$ using the inductive hypothesis that $\gamma \denot{\regr}^{\sharp}_{A} \alpha \le \gamma' \denot{\regr}^{\sharp}_{A'} \alpha'$. For $n = 0$ the inequality to prove reduces to $\gamma \id_A \alpha = A \le A' = \gamma' \id_{A'} \alpha'$. So suppose it is true for $n$:
	\begin{align*}
		\gamma (\denot{\regr}^{\sharp}_{A})^{n+1} \alpha & = \gamma (\denot{\regr}^{\sharp}_{A})^n \denot{\regr}^{\sharp}_{A} \alpha                       & [\text{definition}]          \\
		                                                 & = \gamma (\denot{\regr}^{\sharp}_{A})^n \alpha \gamma \denot{\regr}^{\sharp}_{A} \alpha         & [\gamma \alpha = \id_A]      \\
		                                                 & \le \gamma' (\denot{\regr}^{\sharp}_{A'})^n \alpha' \gamma' \denot{\regr}^{\sharp}_{A'} \alpha' & [\text{inductive hps}]       \\
		                                                 & = \gamma' (\denot{\regr}^{\sharp}_{A'})^{n+1} \alpha'                                           & [\gamma' \alpha' = \id_{A'}]
	\end{align*}
	Now we proceed with the proof of the structural inductive statement:
	\begin{align*}
		\gamma \denot{\regr^\kstar}^{\sharp}_{A} \alpha & = \gamma \left( \bigsqcup_{n \ge 0} (\denot{\regr}^{\sharp}_{A})^n \alpha \right)                    & [\text{definition}]           \\
		                                                & = \gamma \left( \bigsqcup_{n \ge 0} \alpha \gamma (\denot{\regr}^{\sharp}_{A})^n \alpha \right)      & [\alpha \gamma = \id_A]       \\
		                                                & = \gamma \alpha \left( \bigsqcup_{n \ge 0} \gamma (\denot{\regr}^{\sharp}_{A})^n \alpha \right)      & [\alpha \text{ is additive}]  \\
		                                                & \le A \left( \bigsqcup_{n \ge 0} \gamma' (\denot{\regr}^{\sharp}_{A'})^n \alpha' \right)             & [\text{statement above}]      \\
		                                                & \le A' \left( \bigsqcup_{n \ge 0} \gamma' (\denot{\regr}^{\sharp}_{A'})^n \alpha' \right)            & [A \le A']                    \\
		                                                & = \gamma' \left( \bigsqcup_{n \ge 0} \alpha' \gamma' (\denot{\regr}^{\sharp}_{A'})^n \alpha' \right) & [\alpha' \text{ is additive}] \\
		                                                & = \gamma' \left( \bigsqcup_{n \ge 0} (\denot{\regr}^{\sharp}_{A'})^n \alpha' \right)                 & [\alpha' \gamma' = \id_{A'}]  \\
		                                                & = \gamma' \denot{\regr^\kstar}^{\sharp}_{A'} \alpha'                                                 & [\text{definition}]
	\end{align*}
\end{proof}

\begin{proof}[Proof of Theorem~\ref{th:lcla:soundness-rule-simpl}, intensional soundness of rule \lclrule{simplify}]
	Since the proof of Theorem~\ref{th:sota:lcl-soundness} in~\cite{BGGR21} is by induction, we extend it by just proving the new inductive case.

	\proofcase{\lclrule{simplify}}
	(1) It's the same as point (1) of intensional soundness applied to $\lcl{A'}{P}{\regr}{Q}$, since this conclusion doesn't depend on the abstract domain.

	\noindent (2-3)
	\begin{align*}
		A'(Q) & = A(Q)                                             & [\text{hypothesis of the rule}]                       \\
		      & \le A(\denot{\regr} P)                             & [\text{(1) and monotonicity of }A]                    \\
		      & = \gamma \alpha(\denot{\regr} P)                   & [\text{definition}]                                   \\
		      & \le \gamma \denot{\regr}^{\sharp}_{A} \alpha(P)    & [\text{soundness of } \denot{\regr}^{\sharp}_{A}]     \\
		      & \le \gamma' \denot{\regr}^{\sharp}_{A'} \alpha'(P) & [\text{Lemma~\ref{lmm:app:comp-abs-monotone-domain}}] \\
		      & = \gamma' \alpha'(Q)                               & [\text{Soundness of }\lcl{A'}{P}{\regr}{Q}]           \\
		      & = A'(Q)                                            & [\text{definition}]
	\end{align*}
	Hence all the lines are equal; in particular $\gamma \alpha(Q) = \gamma \alpha(\denot{r} P)$ and $\gamma \denot{r}^{\sharp}_{A} \alpha(P) = \gamma \alpha(Q)$. Since $\gamma$ is injective, we get points (2-3) of intensional soundness.
\end{proof}

\begin{proof}[Proof of Theorem~\ref{th:lcla:intrinsic-incompl-simplify}, intrinsic incompleteness of $\LCLA$ with rule \lclrule{simplify}]
	The proof follows closely that of intrinsic incompleteness of $\LCLA$ \cite[Theorem~5.12]{BGGR21}.
	The Turing completeness hypothesis allows to define, for any store $c \in \Sigma$, three regular commands $\regr_{c?}$, $\regr_{\lnot c?}$ and $\regr_{c}$ such that, given any input $S \in C = \pow(\Sigma)$
	\begin{align*}
		\denot{\regr_{c?}} S       & = S \cap \{ c \}      \\
		\denot{\regr_{\lnot c?}} S & = S \setminus \{ c \} \\
		\denot{\regr_{c}} S        & = \{ c \}
	\end{align*}
	Moreover, by Turing completeness we are able to define a regular command $\regr_w$ such that, for all $S \in C$ we have $\denot{\regr_w} S = \emptyset$.
	As an example, for a single variable \code{x} these four commands are
	\begin{align*}
		\regr_w          & \eqdef \code{while (true) \{ skip \}}                    \\
		\regr_{c?}       & \eqdef \code{if (x == c) then skip else } \regr_w        \\
		\regr_{\lnot c?} & \eqdef \code{if (x == c) then }\regr_w \code{ else skip} \\
		\regr_{c}        & \eqdef \code{x := c}
	\end{align*}

	Since $A$ is not trivial, there exists two set of stores $P$ and $R$ such that $P \subsetneq A(P)$ and $A(R) \subsetneq C$. These imply that there exists two stores $a \in A(P) \setminus P$ and $b \in C \setminus A(R)$. Since by monotonicity of $A$ we get $A(\emptyset) \subseteq A(R)$ and we know $b \notin A(R)$, we conclude that $A(\{ b \}) \neq A(\emptyset)$. Moreover, by monotonicity of $A$ we also have $A(\emptyset) \subseteq A(\{ b \})$, so that $A(\emptyset) \subsetneq A(\{ b \})$.
	Given such $a$ and $b$, let us consider the command $\regr$ defined as
	\[
	\regr = (\regr_{a?}; \regr_{b}) \regplus (\regr_{\lnot a?}; \regr_w)
	\]
	for $\regr_{a?}$, $\regr_{b}$, $\regr_{\lnot a?}$ and $\regr_w$ as defined above. We now show that we cannot prove the triple $\lcl{A}{P}{\regr_1; \regr_w}{\emptyset}$ in $\LCLA$ extended with \lclrule{simplify} even though it is intensionally sound.

	First, we verify soundness: $\denot{\regr; \regr_w} P = \denot{\regr_w} (\denot{\regr} P) = \emptyset$, so that $\emptyset \subseteq \denot{\regr; \regr_w} P$, and
	\[
	\alpha(\denot{\regr; \regr_w} P) = \alpha(\emptyset) = \denot{\regr_w}^{\sharp}_{A} (\denot{\regr}^{\sharp}_{A} \alpha(P)) = \denot{\regr; \regr_w}^{\sharp}_{A} \alpha(P)
	\]
	so the triple is intensionally sound.

	However, we can't derive the triple because $\regr$ is not locally complete on $P$ in $A$, nor in any $A' \succeq A$ such that $A(\emptyset) = A'(\emptyset)$, that are the only domain we can simplify to using rule \lclrule{simplify}. The local incompleteness is a consequence of
	\[
	\denot{\regr} A(P) = \denot{\regr_{a?}; \regr_{b}} A(P) \cup \denot{\regr_{\lnot a?}; \regr_w} A(P) = \denot{\regr_{b}} \denot{\regr_{a?}} A(P) \cup \emptyset = \denot{\regr_{b}} \{ a \} = \{ b \}
	\]
	where we used $a \in A(P)$. Using this we easily get $\lnot \complete{A}{P}{\denot{\regr}}$:
	\[
	A(\denot{\regr} P) = A(\emptyset) \subsetneq A(\{ b \}) = A(\denot{\regr} A(P))
	\]

	Formally, to derive the triple $\lcl{A}{P}{\regr; \regr_w}{\emptyset}$, we can apply only three rules: \lclrule{relax}, \lclrule{seq} and \lclrule{simplify}. By the arbitrariness of $P$ we can assume without loss of generality that there is no $P' \subseteq P \subseteq A(P')$, and clearly any $Q' \supseteq \emptyset$ is no longer sound because $\emptyset = \denot{\regr; \regr_w} P$. So we can't apply \lclrule{relax}. Let us distinguish the other two cases.

	\noindent If we apply \lclrule{seq}, we get
	\[
	\infer[(\mathsf{seq})]{\lcl{A}{P}{\regr; \regr_w}{\emptyset}}{\lcl{A}{P}{\regr}{Q} & \lcl{A}{Q}{\regr_w}{\emptyset}}
	\]
	for some intermediate $Q$. However, any provable triple $\lcl{A}{P}{\regr}{Q}$ would imply by soundness (of $\LCLA$ extended with \lclrule{simplify}, Theorem~\ref{th:lcla:soundness-rule-simpl}) local completeness $\complete{A}{P}{\denot{\regr}}$, that is false. So we can't prove the triple starting with an application of \lclrule{seq}

	\noindent If we apply \lclrule{simplify}, we get
	\[
	\infer[(\mathsf{simplify})]{\lcl{A}{P}{\regr; \regr_w}{\emptyset}}{\lcl{A}{P}{\regr; \regr_w}{\emptyset} & A' \succeq A & A'(\emptyset) = A(\emptyset)}
	\]
	for some simplification $A'$ of $A$. Note that we have $A'(\emptyset) \subsetneq A'(\{ b \})$:
	\[
	A'(\emptyset) = A(\emptyset) \subsetneq A(\{ b \}) \subseteq A'(\{ b \})
	\]
	Now we could apply \lclrule{relax} to change $P$ with a $P'$ satisfying $P' \subseteq P \subseteq A'(P')$. However, we have $a \notin P$, hence $a \notin P' \subseteq P$, and $a \in A(P) \subseteq A'(P) = A'(P')$. Now the only rule we can apply is \lclrule{seq}, and as before we need a triple $\lcl{A'}{P'}{\regr}{Q}$ for some $Q$, that would imply $\complete{A'}{P'}{\denot{\regr}}$. But this is not the case, and can be shown with computations as before using that $a \in A'(P') \setminus P'$ to obtain $\denot{\regr} P' = \emptyset$ and $\denot{\regr} A'(P') = \{ b \}$, and the already shown $A'(\emptyset) \subsetneq A'(\{ b \})$.
\end{proof}

\begin{proof}[Proof of Theorem~\ref{th:lcla:clcl-sound-int}]
	The proof follows closely that for LCL (Theorem~\ref{th:lcla:soundness-ext}).\todo{Check the proof for name consistency and references}

	As for $\LCLA$, points (1) and (3) implies point (2):
	\begin{align*}
		\alpha(P) & \le_A \alpha(\bwsem{\regr} Q)              & [\text{(1) and monotonicity of }\alpha]          \\
		          & \le_A \bwsem{\regr}^{\sharp}_{A} \alpha(Q) & [\text{soundness of }\bwsem{\cdot}^{\sharp}_{A}] \\
		          & = \alpha(P)                                & [\text{(3)}]
	\end{align*}
	Therefore, we only have to prove (1) and (3).

	The proof is by induction on the derivation tree of the provable triple $\clcltriple{A}{P}{\regr}{Q}$. We prove here only some relevant cases.\todo{Add all of them?}

	\proofcase{\clclrule{transfer}}
	For (1), we have trivially $\bwsem{\regc} Q \subseteq \bwsem{\regc} Q$. For (2), we have $\alpha(\bwsem{\regc} Q) = \alpha(\bwsem{\regc} \gamma \alpha Q) = \bwsem{\regc}^{\sharp}_A \alpha (Q)$ by $\complete{A}{Q}{\bwsem{\regc}}$ and definition of $\bwsem{\regc}^{\sharp}_A$, respectively (see Fig.~\ref{fig:abs-semantics:bw}).

	\proofcase{\clclrule{relax}}
	For (1) we have
	\begin{align*}
		P & \subseteq P'               & [\text{hp of the rule}]                                     \\
		  & \subseteq \bwsem{\regr} Q' & [\text{inductive hp (1) on } \clcltriple{A}{P'}{\regr}{Q'}] \\
		  & \subseteq \bwsem{\regr} Q  & [\text{hp of the rule}]
	\end{align*}

	For (2), we recall that $P \subseteq P' \subseteq A(P)$ implies $\alpha(P) = \alpha(P')$: one inequality holds by monotonicity of $\alpha$, the other by the adjunctive property of a GC since $A = \gamma \circ \alpha$. Then
	\begin{align*}
		\alpha(P) & = \alpha(P')                            & [\text{hp of the rule } P \subseteq P' \subseteq A(P)]      \\
		          & = \bwsem{\regr}^{\sharp}_{A} \alpha(Q') & [\text{inductive hp (2) on } \clcltriple{A}{P'}{\regr}{Q'}] \\
		          & = \bwsem{\regr}^{\sharp}_{A} \alpha(Q)  & [\text{hp of the rule } Q' \subseteq Q \subseteq A(Q')]
	\end{align*}

	\proofcase{\clclrule{rec}}
	We first observe that
	\begin{align*}
		\bwsem{\regr^{\kstar}} \bwsem{\regr} Q & \le Q \cup \bwsem{\regr^{\kstar}} \bwsem{\regr} Q                                &                                         \\
		                                       & =\bwsem{\regr}^0 Q \cup \bigcup\limits_{n \ge 0} \bwsem{\regr}^n \bwsem{\regr} Q & [\text{Lemma~}\ref{lmm:bwsem-calculus}] \\
		                                       & =\bigcup\limits_{n \ge 0} \bwsem{\regr}^n Q                                      &                                         \\
		                                       & =\bwsem{\regr^{\kstar}} Q                                                        & [\text{Lemma~}\ref{lmm:bwsem-calculus}]
	\end{align*}

	We can then prove (1) by
	\begin{align*}
		P & \subseteq \bwsem{\regr^{\kstar}} (R \cup Q)                            & [\text{inductive hp (1) on } \clcltriple{A}{P}{\regr^{\kstar}}{R \cup Q}] \\
		  & \subseteq \bwsem{\regr^{\kstar}} (\bwsem{\regr} Q \cup Q)              & [\text{inductive hp (1) on } \clcltriple{Q}{R}{\regr}{Q}]                 \\
		  & = \bwsem{\regr^{\kstar}} \bwsem{\regr} Q \cup \bwsem{\regr^{\kstar}} Q & [\text{additivity of }\bwsem{\cdot}]                                      \\
		  & \subseteq \bwsem{\regr^{\kstar}} Q \cup \bwsem{\regr^{\kstar}} Q       & [\text{observation above}]                                                \\
		  & = \bwsem{\regr^{\kstar}} Q
	\end{align*}

	For (2)
	\begin{align*}
		\bwsem{\regr^{\kstar}}^{\sharp}_A \alpha(Q) & \le_A \bwsem{\regr^{\kstar}}^{\sharp}_A (Q \cup R) &                                                                           \\
		                                            & = \alpha(P)                                        & [\text{inductive hp (2) on } \clcltriple{A}{P}{\regr^{\kstar}}{R \cup Q}] \\
		                                            & \le_A \alpha(\bwsem{\regr^{\kstar}} Q)             & [\text{condition (1) shown above}]                                        \\
		                                            & \le_A \bwsem{\regr^{\kstar}}^{\sharp}_A \alpha(Q)  & [\text{soundness of }\bwsem{\cdot}^{\sharp}_A]
	\end{align*}

	\proofcase{\clclrule{iterate}}
	For (1)
	\begin{align*}
		P \cup Q & \subseteq \bwsem{\regr} Q \cup Q                     & [\text{inductive hp (1) on }\clcltriple{A}{P}{\regr}{Q}] \\
		         & = \bwsem{\regr}^1 Q \cup \bwsem{\regr}^0 Q           &                                                          \\
		         & \subseteq \bigcup\limits_{n \ge 0} \bwsem{\regr}^n Q &                                                          \\
		         & = \bwsem{\regr^{\kstar}} Q                           & [\text{Lemma~}\ref{lmm:bwsem-calculus}]
	\end{align*}

	For (2), we first prove by induction that $(\bwsem{\regr}^{\sharp}_{A})^n \alpha(Q) \le_A \alpha(Q)$. Recall that, by the adjunctive property of a Galois connection, $P \subseteq A(Q)$ iff $\alpha(P) \le_A \alpha(Q)$. The base case $n = 0$ is trivial because $(\bwsem{\regr}^{\sharp}_{A})^0$ is the identity. Suppose it holds for some $n$: then
	\begin{align*}
		(\bwsem{\regr}^{\sharp}_{A})^{n+1} \alpha(Q) & = \bwsem{\regr}^{\sharp}_{A} (\bwsem{\regr}^{\sharp}_{A})^n \alpha(Q) &                                                          \\
		                                             & \le_A \bwsem{\regr}^{\sharp}_{A} \alpha(Q)                            & [\text{inductive hp}]                                    \\
		                                             & = \alpha(P)                                                           & [\text{inductive hp (2) on }\clcltriple{A}{P}{\regr}{Q}] \\
		                                             & \le_A \alpha(Q)                                                       & [\text{hp of the rule } P \le A(Q)]
	\end{align*}

	From this, we observe that
	\[
	\alpha(Q) = (\bwsem{\regr^{\kstar}}^{\sharp}_{A})^0 \alpha(Q) \le_A \bigcup\limits_{n \ge 0} (\bwsem{\regr^{\kstar}}^{\sharp}_{A})^n \alpha(Q) \le_A \bigcup\limits_{n \ge 0} \alpha(Q) = \alpha(Q)
	\]
	and therefore $\bwsem{\regr^{\kstar}}^{\sharp}_{A} \alpha(Q) = \alpha(Q)$ by Lemma~\ref{lmm:bwsem-calculus}.

	We then conclude the proof of (2) with
	\begin{align*}
		\bwsem{\regr^{\kstar}}^{\sharp}_{A} \alpha(Q) & = \alpha(Q)                  &                                     \\
		                                              & = \alpha(P) \vee_A \alpha(Q) & [\text{hp of the rule } P \le A(Q)] \\
		                                              & = \alpha(P \cup Q)           & [\text{additivity of } \alpha]
	\end{align*}
\end{proof}

\begin{sidewaysfigure}
	\centering
	{
		\begin{math}
			\infer[\clclrule{seq}]
			{\clcltriple{\Oct}{T_{2M}}{\regr_w}{R_{2M}}}
			{
				\infer{\clcltriple{\Oct}{T_{2M}}{\code{n > 0?}}{x + n = \bar{k}}}{\complete{\Oct}{x + n = \bar{k}}{\bwsem{\code{n > 0?}}}}
				&
				\infer{\clcltriple{\Oct}{x + n = \bar{k}}{\code{x := x + n}}{x = \bar{k}}}{\complete{\Oct}{x = \bar{k}}{\bwsem{\code{x := x + n}}}}
				&
				\infer{\clcltriple{\Oct}{x = \bar{k}}{\code{n := nondet()}}{R_{2M}}}{\complete{\Oct}{R_{2M}}{\bwsem{\code{n := nondet()}}}}
			}
		\end{math}
	}

	\bigskip
	{
		\begin{math}
			\infer[\clclrule{seq}]
			{\clcltriple{\Oct}{S_{2M}}{\regr_w}{T_{2M} \lor R_{2M}}}
			{
				\infer{\clcltriple{\Oct}{S_{2M}}{\code{n > 0?}}{x + n \le \bar{k}}}{\complete{\Oct}{x + n \le \bar{k}}{\bwsem{\code{n > 0?}}}}
				&
				\infer{\clcltriple{\Oct}{x + n \le \bar{k}}{\code{x := x + n}}{x \le \bar{k}}}{\complete{\Oct}{x \le \bar{k}}{\bwsem{\code{x := x + n}}}}
				&
				\infer{\clcltriple{\Oct}{x < \bar{k} \lor x = \bar{k}}{\code{n := nondet()}}{T_{2M} \lor R_{2M}}}{\complete{\Oct}{T_{2M} \lor R_{2M}}{\bwsem{\code{n := nondet()}}}}
			}
		\end{math}
	}

	\bigskip
	{
		\begin{math}
			\infer[\clclrule{rec}]
			{\clcltriple{\Oct}{S_{2M} \lor R_{2M}}{\regr_w^{\kstar}}{R_{2M}}}
			{
				\infer[\clclrule{iterate}]
				{\clcltriple{\Oct}{S_{2M} \lor R_{2M}}{\regr_w^{\kstar}}{T_{2M} \lor R_{2M}}}
				{
					S_{2M} \subseteq \Oct(T_{2M} \lor R_{2M})
					& \infer{\clcltriple{\Oct}{S_{2M}}{\regr_w}{T_{2M} \lor R_{2M}}}{\dots}
				}
				& \infer{\clcltriple{\Oct}{T_{2M}}{\regr_w}{R_{2M}}}{\dots}
			}
		\end{math}
	}

	\caption{Derivation of the CLCL triple $\clcltriple{\Oct}{(n > 0 \land x + n \le 2000000) \lor R_{2M}}{\regr_w^{\kstar}}{R_{2M}}$. For brevity, we omit rule name \clclrule{transfer}, we let $\bar{k} \eqdef 2000000$, we recall that $R_{2M} \eqdef (x = \bar{k} \land n \le 0)$, $T_{2M} \eqdef (x + n = \bar{k} \land n > 0)$ and we define $S_{2M} \eqdef (x + n \le \bar{k} \land n > 0)$.}
	\label{fig:app:clcl-derivation}
\end{sidewaysfigure}
