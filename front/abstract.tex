\thispagestyle{plain}
\begin{center}
	\Large
	\textbf{Abstract}
\end{center}
Formal static analysis has always been mainly focused on over-approximation to prove correctness. The dual approach of under-approximation is instead best suited to prove incorrectness. While the two paradigms can be applied separately, it has been shown that their are able to cooperate to decide more effectively both correctness and incorrectness.
Abstract interpretation is a general framework for static analyses, that has been successfully applied to many fields. Given its flexibility, we are interested in using it to explain and define static analyses which combines over- and under-approximation.
In this proposal, we survey some works which already exploit both, present some our results in this direction, and outline how we may extend them further.
