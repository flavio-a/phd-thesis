\thispagestyle{plain}
\begin{center}
	\Large
	\textbf{Abstract}
\end{center}
Formal static analysis has always been mainly focused on over-approximation to prove correctness. Recently, the dual approach of under-approximation has gained attention as a formal basis to prove incorrectness. While the two paradigms can be applied separately, it has been shown that their are able to cooperate to decide more effectively both correctness and incorrectness.
In this thesis, we study analogies and differences between over and under-approximation to understand what can and cannot be ported from the well studied over-approximation theory to the less studied under-approximation one, and try to combine the two approaches to improve both.
We first focus on abstract interpretation, finding limitations in approaches based on under-approximating abstract domains. We then turn our attention to program logics, classifying known results as over or under-approximation of forward or backward semantics, and defining a new backward-oriented proof system for backward under-approximation. Then we present novel extensions of two approaches combining over and under-approximation in non-trivial ways, namely Local Completeness Logic and Property Directed Reachability, showing how they improve on proving both correctness and incorrectness of the system under analysis.
