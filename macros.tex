%%%%%%%%%%%%%%%%%%%% COMMANDS
\DeclarePairedDelimiter{\denot}{\llbracket}{\rrbracket}
\DeclarePairedDelimiter{\edenot}{\llparenthesis}{\rrparenthesis}
\DeclarePairedDelimiter{\abs}{\lvert}{\rvert}

\newcommand{\code}[1]{\texttt{#1}}
\newcommand{\labelitem}[1]{\item[] \textit{#1}:}
\newcommand{\Var}{\text{Var}}

%%%%%%%%%%%%%%%%%%%% THEOREMS
\theoremstyle{plain}
\newtheorem{theorem}{Theorem}
\counterwithin{theorem}{chapter}

\newtheorem{corollary}[theorem]{Corollary}
\providecommand*{\corollaryautorefname}{Corollary}

\newtheorem{lemma}[theorem]{Lemma}
\providecommand*{\lemmaautorefname}{Lemma}

\newtheorem{prop}[theorem]{Proposition}
\providecommand*{\propautorefname}{Proposition}

\newcounter{direction}

%\declaretheorem[style=remark,qed=$\blacksquare$,sibling=theorem]{example}
%\declaretheorem[style=definition,sibling=theorem]{definition}
%\declaretheorem[style=plain,numbered=no]{assumption}
%\declaretheorem[style=definition,sibling=direction]{direction}

%\declaretheorem[style=definition,sibling=theorem]{definition}
%\declaretheorem[style=plain,numbered=no]{assumption}

%\newtheorem{remark}[theorem]{Remark}
%\providecommand*{\remarkautorefname}{Remark}

%%%%%%%%%%%%%%%%%%%% MATH SYMBOLS
%\newcommand{\setN}{\mathbb{N}}
\newcommand{\setZ}{\mathbb{Z}}
%\newcommand{\katR}{\mathcal{R}}
\newcommand{\svert}{\,\vert\,}
\newcommand{\sdot}{\,.\,}
\newcommand{\pow}{\mathcal{P}}
\newcommand{\expe}{\mathsf{e}}
\newcommand{\regr}{\mathsf{r}}
\newcommand{\regc}{\mathsf{c}}
%\newcommand{\regw}{\mathsf{w}}
\newcommand{\Reg}{\mathsf{Reg}}
\newcommand{\Imp}{\mathsf{Imp}}
\newcommand{\Exp}{\mathsf{Exp}}
\newcommand{\regtimes}{\otimes}
\newcommand{\regplus}{\oplus}
\newcommand{\kstar}{\star}
\newcommand{\eqdef}{\triangleq}

%%%%%%%%%%%%%%%%%%%% ORDER-THEORETIC SYMBOLS
\newcommand{\fix}{\text{fix}}
\newcommand{\id}{\text{id}}
\newcommand{\lfp}{\text{lfp}}
\newcommand{\op}{\text{op}}
\newcommand{\image}{\text{Im}}

%%%%%%%%%%%%%%%%%%%% ABSINT
\newcommand{\Abs}{\text{\normalfont Abs}}
\newcommand{\gc}[4]{\langle #1, (#2, #3), #4 \rangle}
%\newcommand{\gi}[3]{\langle #1, (#2, \id_{#3}), #3 \rangle}
\newcommand{\complete}[3]{\mathbb{C}^{#1}_{#2}(#3)}
\newcommand{\Int}{\text{Int}}
\newcommand{\Oct}{\text{Oct}}
\newcommand{\Moore}{\mathcal{M}}
%\newcommand{\poly}{\text{poly}}
%\newcommand{\drop}{\text{drop}}
%\newcommand{\concat}{\text{concat}}
%\newcommand{\macheps}{\textbf{u}}
%\newcommand{\fp}{\text{fp}}

%%%%%%%%%%%%%%%%%%%% Logics (HL, IL, LCL)
\newcommand{\underexact}[1]{\ensuremath [#1]}
\newcommand{\overexact}[1]{\ensuremath \{#1\}}
\newcommand{\undertriple}[3]{\underexact{#1}~#2~\underexact{#3}}
\newcommand{\overtriple}[3]{\overexact{#1}~#2~\overexact{#3}}
\newcommand{\bdiamond}[1]{\langle #1 \vert}
\newcommand{\spost}{\textbf{sp}}
\newcommand{\test}{\text{test}}
\newcommand{\lcl}[4]{\vdash_{#1}\nobreak\undertriple{#2}{#3}{#4}}
\newcommand{\lrule}[1]{$(\mathsf{#1})$}
\newcommand{\LCLA}{\text{LCL}_A}

%%%%%%%%%%%%%%%%%%%% PDR
\newcommand{\xbar}{\overline{x}}
\newcommand{\clause}{\text{clause}}
\newcommand{\states}{\mathcal{S}}
\newcommand{\true}{\textbf{true}}
\newcommand{\timplies}{\Rightarrow}
